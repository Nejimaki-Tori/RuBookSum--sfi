\documentclass{superfri}
\usepackage[utf8]{inputenc}
\usepackage[T1]{fontenc}
\usepackage{amsmath}
\usepackage{amssymb}
\usepackage{amsthm}
\usepackage[cal=rsfso]{mathalfa}
\usepackage{booktabs}
\usepackage{multirow}
\usepackage{makecell}
\usepackage{float}
\usepackage{algorithm}
\usepackage{algpseudocode}
\usepackage[normalem]{ulem}
\usepackage{graphicx}
\usepackage{lmodern}
\usepackage{xurl} 
\usepackage[hidelinks]{hyperref}
\usepackage{type1cm}

\renewcommand{\algorithmicrequire}{\textbf{Input:}}

\begin{document}
\raggedbottom
\author{D.\,A.~Grigoriev\footnote{\label{msu}Lomonosov Moscow State University, Moscow Center for Fundamental and Applied Mathematics, \\ Moscow, Russian Federation}\footnote{E-mail: dagrig14@yandex.ru} \and D.\,V.~Khudiakov\footnoteref{msu}\footnote{E-mail: hydikovv17914@gmail.com} \and D.\,I.~Chernyshev\footnoteref{msu}\footnote{E-mail: chdanorbis@yandex.ru}}

\title{RuBookSum: a dataset for Russian literature abstractive summarization}

\maketitle{}

\begin{abstract}
The majority of existing Russian document summarization datasets focus on short-form source documents which doesn't require complex causal analysis or 
coreference resolutions. Furthermore, processing longer multi-page texts pose a serious challenge to current generation of language models as the 
limited context window complicates response generation by demanding additional task partitioning. 
To lay the groundwork for future research of the problem we introduce RuBookSum, an abstractive summarization dataset for Russian long-form narrative summarization. 
Our dataset covers documents from various literature domains, including fiction, classic, children books and popular science, and include high-quality human-written 
summaries. To establish a baseline we evaluate popular open-source large language models and provide comprehensive analysis on their performance. Additionally, 
we propose an optimized algorithms for long-document summarization which enable up to 300\% summary generation speed up without significant drops in quality.

\keywords{LLM, summarization, literature, books, brief retelling}
\end{abstract}



% ================= INTRODUCTION =================

\section*{Introduction}
Automatic text summarization is one of the key tasks in natural language processing. The goal is to create an informative synopsis of the source text while preserving its main meaning.
In recent years, with the advent of large language models (LLMs), interest in automating summarization has increased across many genres, including fiction.
Unlike scientific, news, or technical texts, fiction is characterized by high stylistic and semantic complexity.
Non-linear storytelling, imagery, metaphor, and stylistic devices make synopsis writing especially challenging.
The limited context window of modern models further complicates processing long texts as it imposes additional text generation constraints and demands additional task partitioning.

At present moment there are not many datasets focusing specifically on summarizing fiction, and available collections focus on non-Russian material.
BookSum \cite{BookSum} is one of the first and best-known English-language datasets for abstractive summarization of narrative works. 
It contains books, plays, and short stories paired with summaries of varying granularity (paragraph level, chapter level, book level).
Echoes from Alexandria \cite{alexandria} is a multilingual corpus of fiction, including five languages: English, German, French, Italian, and Spanish.
FABLES \cite{fables} is a hand-curated corpus designed to evaluate factual faithfulness of summaries for book-length fiction.
It includes 3,158 claims extracted from LLM-generated summaries for 26 books.
Each claim is evaluated across model outputs by experts.
According to FABLES, even advanced models (e.g., Claude) commit 20–30\% factual errors, including distorted
causal relations, incorrect characterization of protagonists, and overemphasis on minor details,
judged by three criteria: agreement with original events, logical correctness, and absence of distortions.

To study the specifics of  Russian long-narrative automatic summarization we introduce RuBookSum, a dataset for Russian literature abstractive summarization. 
Our dataset contains high-quality human-written summaries for document of different domains including fiction, popular science, children books and classical literature.
To demonstrate the issues of multi-page text summarization we conduct an extensive evaluation of popular open-source large language models. 
Our analysis finds that only largest models exceeding 100 billion parameters are able to fully comprehend long-range causal relations while smaller models 
only capture general semantics. Additionally, to adapt the models for the task we propose new abstractive summarization algorithms optimized for long-document processing. 
Compared to existing approaches new methods achieve up to 300\% summary generation speed up while retaining the same level of quality.

Code and data are publicly available\footnote{\label{git}\url{https://github.com/Nejimaki-Tori/RuBookSum}}.

% ================= DATASET =================

\section{Dataset}
At the moment of the study, there were no publicly available corpora designed specifically for Russian literature summarization.
To address the lack of resources new dataset was created, using "Narodny Briefly" platform \cite{Briefly} where users publish summaries for popular books.
The summaries vary in length-from a few sentences to several paragraphs-and in style: some reproduce key phrases verbatim, 
while others use free-form narration. Some cover the whole work, others split content by chapter.
Usually they contain the main facts and conclusions from the source text, but may include the author’s commentary.

To collect the respective summary sources we leveraged Librusec digital library \cite{librusec}, one of the largest Russian-language online book collections.
Each text underwent automatic preprocessing: meta-information (e.g., titles, chapter descriptions, technical inserts) were removed,
then the text was formatted into a unified, standardized form suitable for use with models.

To better link books with their summaries, cosine similarity was used: the author name text from Briefly \cite{Briefly} and from Librusec \cite{librusec} was embedded via SentenceTransformers
with the model\footnote{\label{encoder}\url{https://huggingface.co/deepvk/USER-bge-m3}}
and compared using cosine similarity.
The summaries were automatically cleaned of HTML tags, comments, and service markers using LLM Meta-\allowbreak Llama 3-\allowbreak 70B-\allowbreak Instruct.
Then Librusec was searched and a collection of "book text – summary" pairs was formed.
\\The resulting dataset includes:
\begin{itemize}
  \item 600+ cleaned user summaries from "Narodny Briefly" \cite{Briefly};
  \item 40+ different genres;
  \item source works from the Librusec digital library \cite{librusec}.
\end{itemize}

\tab{tab:dataset_info}{Dataset overview}{
\begin{tabular}{l|cccc}
\hline
\textbf{Dataset} & \makecell{Number of \\ documents} & \makecell{Avg. document \\ length \\ (\# words)} & \makecell{Avg. summary \\ length \\ (\# words)} & \makecell{Compression ratio \\ (summary length \\/ text length)} \\
\hline
\textbf{RuBookSum}  & 634 & 35052.64 & 700.77 & 8.43\% \\
\hline
BookSum    & 405 & 112885.15 & 1167.20 & 0.79\% \\
Gazeta     & 60964 & 632.77 & 41.94 & 6.99\% \\
\hline
\end{tabular}
}

\fig{width=0.8\linewidth}{figures/genres_pie.png}{Distribution of texts by genres (top 10 genres)}

\figref{figures/genres_pie.png} shows the genre distribution in the collection and \tabref{tab:dataset_info} gives dataset statistics versus analogs.

% ================= METHODOLOGY =================
\section{Methodology}
\subsection{Hierarchical method}
Most common method (Algorithm \ref{alg:h}) for long-document summarization \cite{hierarchical} splits the text into chunks and
generates a local summary for each chunk.
These chunks are then grouped, and summaries are merged into higher-level summaries.
Thus a summary hierarchy is created the last level of which (tree root) yields the final book-level summary.

\subsection{Node filtering optimization}
The classical hierarchical method constructs the final summary through a multi-layered combination of intermediate summaries derived from individual text chunks.
However, literature often contain chunks that have little impact on plot development or contain redundant information without any additional details. 
During the generation of the final summary, these chunks can shift the narrative towards repetitive content thus reducing the overall informativeness.

To address this issue, a node filtering mechanism based on cosine similarity was implemented (Algorithm \ref{alg:h_f}).
To eliminate low-informative or redundant chunks, we evaluate cosine similarity between all intermediate summaries at each hierarchy level.
Chunks that are close in cosine similarity to previous ones are considered redundant are excluded from compilation of the summary at the current level.
This modification aims to accelerate generation by removing potentially superfluous parts of information, 
thereby increasing the salient detail density in the final summaries.

\noindent
\begin{minipage}[t]{0.49\textwidth}
  \begin{algorithm}[H]
  \caption{Hierarchical method}\label{alg:h}
    \begin{algorithmic}
      \Require $W$ - model context window, $D$ - input text of length $L \gg W$, $p_\theta$ - model, $C$ - chunk length
      \State Split $D$ into chunks $c_1\dots c_{\lceil \frac{L}{C} \rceil}$
      \State $\ell \gets 0$
      \State $S_\ell \gets \{c_1\dots c_{\lceil \frac{L}{C} \rceil}\}$
      \Repeat
        \State $Groups \gets GroupSummaries(S_\ell)$
        \State $\ell \gets \ell + 1$
        \For{$g \in Groups$}
          \State $S_\ell \gets \{MergeGroup(p_\theta,g)\}$
        \EndFor
      \Until{$|S_\ell|=1$}
      \State \Return $S_\ell[0]$
    \end{algorithmic}
  \end{algorithm}
\end{minipage}\hfill
\begin{minipage}[t]{0.49\textwidth}
  \begin{algorithm}[H]
    \caption{Hierarchical method with node filtering}\label{alg:h_f}
    \begin{algorithmic}
      \Require $W$ - model context window, $D$ - input text of length $L \gg W$, $p_\theta$ - model, $\theta$ - similarity threshold, $C$ - chunk length
      \State Split $D$ into chunks $c_1\dots c_{\lceil \frac{L}{C} \rceil}$
      \State $\ell \gets 0$
      \State $S_\ell \gets \{c_1\dots c_{\lceil \frac{L}{C} \rceil}\}$
      \Repeat
        \For{$s_i \in S_\ell$}
          \State $e_i \gets Encoder(s_i)$
          \State $M_{ij} \gets \dfrac{\mathbf{e}_i\cdot \mathbf{e}_j}{\|\mathbf{e}_i\|\,\|\mathbf{e}_j\|}$\Comment{Embedding matrix}
          \State Compute the maximum similarity
        \State with previous summaries.
          \State $m_j=\max_{i<j} M_{ji}$
          \State $S_\ell \gets \{s_i \mid m_i < \theta$ \textbf{ or } $i = 0 \}$
        \EndFor
        \State $Groups \gets GroupSummaries(S_\ell)$
        \State $\ell \gets \ell + 1$
        \For{$g \in Groups$}
          \State $S_\ell \gets \{MergeGroup(p_\theta,g)\}$
        \EndFor
      \Until{$|S_\ell|=1$}
      \State \Return $S_\ell[0]$
    \end{algorithmic}
  \end{algorithm}
\end{minipage}

\vspace{0.4em}
\subsection{Text-Blueprint}
This method \cite{blueprint} is essentially a modification of the hierarchical method 
that improves summary robustness by building an intermediate outline before text generation  (Algorithm \ref{alg:bl}).
The outline is formed as a set of question-answer pairs, which enhances the controllability of the generation process and ensures the structured nature of the result.
First the model creates a list of questions reflecting key events, themes, and characters. Then short answers are automatically generated for each question.
This structure serves as a blueprint used to produce the final summary.


\subsection{Question clustering optimization}
The baseline blueprint implementation generates a question-answer outline for each chunk and at each merge level. 
With fiction, however, questions produced for different chunks may overlap and yield conflicting answers, which in turn corrupts merging process, 
making the summary less structured and complete. 
Moreover, generating an outline at every step slows the method and consumes extra computational time.
To address the issue we add additional question clustering step aimed at reducing merge level content overlap  (Algorithm \ref{alg:bl_cl}). 
The obtained question clusters are generalized using the same summary generation LLM to produce universal question-answer outline.

\noindent
\begin{minipage}[t]{0.49\textwidth}
  \begin{algorithm}[H]
    \caption{Blueprint method}\label{alg:bl}
    \begin{algorithmic}
      \Require $W$ - model context window, $D$ - input text of length $L \gg W$, $p_\theta$ - model, $C$ - chunk length, $R$ - length limit
      \State Split $D$ into chunks $c_1\dots c_{\lceil \frac{L}{C} \rceil}$
      \State $\ell \gets 0$
      \State $S_\ell \gets \{c_1\dots c_{\lceil \frac{L}{C} \rceil}\}$
      \Repeat \Comment{Merging summaries}
        \State $Groups \gets GroupSummaries(S_\ell)$
        \State $\ell \gets \ell + 1$
        \For{$g \in Groups$}
          \If{$Length(g) > R$}
            \State $b_i \gets GenerateBlueprint(p_\theta, g)$\
            \State $S_\ell \gets S_\ell \cup \{SumWithBp(p_\theta, b_i, g)\}$\
          \Else
            \State $S_\ell \gets \{g\}$\
          \EndIf
        \EndFor
      \Until{$|S_\ell|=1$}
      \State \Return $S_\ell[0]$
    \end{algorithmic}
  \end{algorithm}
\end{minipage}\hfill
\begin{minipage}[t]{0.49\textwidth}
  \begin{algorithm}[H]
  \caption{Blueprint method with clustering}\label{alg:bl_cl}
      \begin{algorithmic}
        \Require $W$ - model context window, $D$ - input text of length $L \gg W$, $p_\theta$ - model, $C$ - chunk length, $R$ - length limit
        \State Split $D$ into chunks $c_1\dots c_{\lceil \frac{L}{C} \rceil}$
        \State $\ell \gets 0$
        \State $S_\ell \gets \{c_1\dots c_{\lceil \frac{L}{C} \rceil}\}$
        \For{$c \in S_\ell$}
          \State $b \gets GenerateBlueprint(p_\theta, c)$
          \State $Q' \gets \{ExtractQuestions(p_\theta, b)\}$
        \EndFor
        \State $K \gets Clusterize(Q')$
        \For{$k_i \in K$}
          \State $Q  \gets Q \cup \{Generalize(p_\theta, k_i)\}$
        \EndFor
        \Repeat \Comment{Merging summaries}
        \State $Groups \gets GroupSummaries(S_\ell)$
        \State $\ell \gets \ell + 1$
        \For{$g \in Groups$}
          \State $S_\ell  \gets S_\ell \cup \{SumWithBp(p_\theta, Q, g)\}$\
        \EndFor
      \Until{$|S_\ell|=1$}
      \State \Return $S_\ell[0]$
      \end{algorithmic}
  \end{algorithm}
\end{minipage}

% ================= METRICS =================

\section{Metrics}

For an objective comparison of the described methods and models in the task of literature summarization, four metrics were considered.

\textbf{ROUGE-L} \cite{rouge} - based on the length of the longest common subsequence (LCS) between the generated summary $S$ and the reference $R$.
\begin{equation}
  \text{Precision} = \frac{\mathrm{LCS}(S,R)}{|S|},
\end{equation}
\begin{equation}
  \text{Recall} = \frac{\mathrm{LCS}(S,R)}{|R|},
\end{equation}
\begin{equation}
  \text{ROUGE-L} = \frac{2\;\text{Precision}\;\cdot\;\text{Recall}}{\text{Precision} + \text{Recall}},
\end{equation}

\textbf{BERTScore} \cite{bertscore}. For every token pair from prediction and reference we compute cosine similarity of their embeddings in. Then:
\begin{equation}
P = \frac{1}{|S|}\sum_{t\in S}\max_{u\in R}\!\mathrm{sim}(e_t, e_u),
\end{equation}

\begin{equation}
R = \frac{1}{|R|}\sum_{u\in R}\max_{t\in S}\!\mathrm{sim}(e_u, e_t),
\end{equation}

\begin{equation}
\text{BERTScore} = \frac{2\,P\,R}{P+R},
\end{equation}

where \(S\) is the reference text and \(R\) is the generated one.

\textbf{Key question coverage (Coverage)} - the proportion of questions that are answered in reference that are covered in generated summary:
\begin{equation}
  \text{Coverage} = \frac{\#\{q_i\colon P(\text{yes}\mid q_i,S)\!>\!0.75\}}{N},
\end{equation}
where $N$ is the total number of questions and $P(\text{yes}\mid q_i,S)$ is the probability that the answer to $q_i$ is present in $S$,
obtained with an LLM.

\textbf{Factual agreement (Agreement)} - the average cosine similarity between answers $a_i^{\text{pred}}$ generated based on predicted summary and answers $a_i^{\text{ref}}$ obtained from reference summary to the same questions from Coverage metric:
\begin{equation}
  \text{Agreement} = \frac{1}{N}\sum_{i=1}^N \mathrm{sim}\bigl(a_i^{\text{pred}},\,a_i^{\text{ref}}\bigr),
\end{equation}
where \texttt{sim} is cosine similarity of embeddings.

% ================= EXPERIMENTAL SETUP =================

\section{Experimental setup}

All measurements were performed on the test split of the dataset.
For all methods the summaries were limited to 500 words maximum.
The input text was split into fixed-\allowbreak size chunks of 2,000 tokens which were obtained by \texttt{AutoTokenizer} from \texttt{DeepPavlov/rubert-\allowbreak base-\allowbreak cased} \footnote{\label{pavlov}\url{https://huggingface.co/DeepPavlov/rubert-base-cased}} with default settings.
To ensure reproducibility the random seed is fixed ($random\_seed = 42$). 
To obtain embeddings we used USER-bge-m3 model\footnoteref{encoder}. To generate questions and answers for Coverage and Agreement metrics we use Qwen3-235B-A22B \cite{qwen3} model. 

In the \textbf{hierarchical method with node filtering}, cosine similarity threshold is set at $\theta=0.85$: if for a summary $S_j$ there existed a previous summary $S_i$ with a cosine similarity above this threshold, then $S_j$ is discarded as redundant. 
This choice of threshold provides a compromise between preserving meaningful information and eliminating duplication, which empirically led to a noticeable reduction in the volume of intermediate representations without significant degradation in quality.

In the \textbf{blueprint method} with question clustering, we utilize KMeans clustering algorithm. The number of clusters is chosen using a heuristically derived rule:
\begin{equation}
  n_{\text{clusters}} = \max\!\left(2,\; \left\lceil \sqrt{N_{\text{questions}}} \right\rceil\right)
\end{equation}
where $N_{\text{questions}}$ is the total number of questions generated across all chunks before clustering.

Runtime was measured as the average value (in seconds) of the generation time per book for each method across 100 books.

% ================= RESULTS =================

\section{Results}

\subsection{Models used}
The experiments used the following large language models:
RuadaptQwen2.5-7B-Lite-Beta \cite{ruadapt},
RuadaptQwen3-32BInstruct-v2 \cite{ruadapt},
DeepSeek V3 \cite{deepseek},
Qwen3-235B-A22B \cite{qwen3},
tpro \cite{tpro} and yagpt5lite \cite{yagpt}.
In all tables, models are grouped by size, and the best results within each parameter group are highlighted.

\tab{tab:results_models}{Results by methods and models}{
\centering
\small
\setlength{\tabcolsep}{2pt}
  \begin{tabular}{l|l|c|c|c|c}
    \hline
    Model & Metrics &
    \makecell{Hierarchical} &
    \makecell{Blueprint} &
    \makecell{Hierarchical \\ with node filtering} &
    \makecell{Blueprint \\ with clustering} \\
    \hline
    \multirow{5}{*}{DeepSeek V3}
    & bertscore  & \uline{60.0 ± 3.1} & 58.0 ± 4.0 & 60.0 ± 2.9 & 58.4 ± 3.6 \\
    & rouge-l    & 13.7 ± 3.9 & 12.6 ± 4.6 & 13.5 ± 3.7 & 11.2 ± 3.9 \\
    & coverage   & \uline{\textbf{53.57 ± 21.66}} & 40.19 ± 23.68 & \uline{\textbf{45.00 ± 23.03}} & \uline{\textbf{34.68 ± 23.77}} \\
    & agreement & \uline{42.38 ± 17.73} & 32.31 ± 19.33 & 35.64 ± 18.88 & \uline{27.76 ± 19.75} \\
    & time       & 196.77 ± 187.85 & 315.67 ± 321.89 & 147.21 ± 146.4 & 132.60 ± 197.25 \\
    \cline{2-6}
    \multirow{5}{*}{Qwen3-235B-A22B}
    & bertscore  & 61.2 ± 3.0 & \uline{61.6 ± 3.3} & \uline{60.9 ± 2.7} & \uline{59.3 ± 3.4} \\
    & rouge-l    & \uline{14.9 ± 4.0} & \uline{15.8 ± 4.5} & \uline{14.8 ± 3.7} & \uline{12.2 ± 3.6} \\
    & coverage   & 52.48 ± 20.79 & \uline{\textbf{54.78 ± 21.16}} & 44.54 ± 23.03 & 30.19 ± 21.96 \\
    & agreement & 41.68 ± 17.18 & \uline{43.99 ± 17.54} & \uline{35.67 ± 18.87} & 24.10 ± 17.62 \\
    & time       & 103.49 ± 97.30 & 230.35 ± 271.03 & 83.06 ± 102.05 & 158.30 ± 196.35 \\
    \hline
    \multirow{5}{*}{\makecell{RuadaptQwen3-32B\\Instruct-v2}}
    & bertscore  & 57.3 ± 2.9 & 58.9 ± 3.6 & 57.7 ± 3.3 & 55.3 ± 3.3 \\
    & rouge-l    & 11.0 ± 2.4 & 10.6 ± 3.2 & 10.7 ± 2.4 & 7.8 ± 2.1 \\
    & coverage   & 33.12 ± 21.50 & 33.18 ± 22.83 & 32.19 ± 22.52 & 17.72 ± 15.23 \\
    & agreement & 25.25 ± 16.94 & 26.21 ± 18.22 & 24.82 ± 17.74 & 13.97 ± 12.39 \\
    & time       & 218.30 ± 195.16 & 379.24 ± 500.40 & 166.79 ± 164.61 & 286.35 ± 395.97 \\
    \cline{2-6}
    \multirow{5}{*}{tpro}
    & bertscore  & \uline{59.4 ± 3.0} & \uline{59.0 ± 4.9} & \uline{59.5 ± 3.3} & \uline{58.2 ± 3.7} \\
    & rouge-l    & \uline{13.8 ± 3.1} & \uline{14.7 ± 4.9} & \uline{13.5 ± 3.0} & \uline{11.8 ± 3.9} \\
    & coverage   & \uline{40.27 ± 20.23} & \uline{40.83 ± 22.42} & \uline{37.13 ± 20.72} & \uline{26.03 ± 18.44} \\
    & agreement & \uline{31.77 ± 16.63} & \uline{32.60 ± 18.57} & \uline{29.44 ± 16.83} & \uline{20.83 ± 15.26} \\
    & time       & 367.32 ± 324.49 & 592.39 ± 772.19 & 267.73 ± 253.34 & 247.59 ± 361.20 \\
    \hline
    \multirow{5}{*}{\makecell{RuadaptQwen2.5-7B\\Lite-Beta}}
    & bertscore  & 55.4 ± 2.9 & 56.1 ± 4.9 & 55.8 ± 2.9 & 54.0 ± 4.0 \\
    & rouge-l    & 8.6 ± 2.5 & 10.1 ± 3.9 & 8.7 ± 2.5 & 7.7 ± 2.8 \\
    & coverage   & 19.66 ± 17.77 & 24.94 ± 21.08 & 20.31 ± 17.95 & 15.51 ± 14.83 \\
    & agreement & 15.16 ± 14.11 & 20.03 ± 17.50 & 15.94 ± 14.39 & 12.23 ± 12.30 \\
    & time       & 68.86 ± 64.85 & 126.84 ± 145.74 & 53.59 ± 47.28 & 76.66 ± 91.78 \\
    \cline{2-6}
    \multirow{5}{*}{yagpt5lite}
    & bertscore  & \uline{62.5 ± 3.5} & \uline{61.1 ± 3.8} & \uline{62.1 ± 3.2} & \uline{61.5 ± 3.3} \\
    & rouge-l    & \uline{16.9 ± 5.1} & \uline{15.8 ± 5.1} & \uline{16.4 ± 4.7} & \uline{14.3 ± 4.4} \\
    & coverage   & \uline{36.85 ± 19.40} & \uline{33.17 ± 21.58} & \uline{31.75 ± 20.06} & \uline{24.28 ± 16.95} \\
    & agreement & \uline{29.69 ± 16.43} & \uline{26.58 ± 18.13} & \uline{25.60 ± 16.85} & \uline{19.70 ± 14.29} \\
    & time       & 31.02 ± 28.51 & 113.34 ± 123.78 & 27.39 ± 28.05 & 42.15 ± 56.50 \\
    \hline
  \end{tabular}
}

\subsection{Findings}
\tabref{tab:results_models} shows metrics of automatic book summarization across models and methods. 
The best overall performance was achieved by Qwen3-235B-A22B: it delivered the highest coverage and answer agreement.
At the same time, the hierarchical method with node filtering offered the best quality–time trade-off. It significantly sped up processing 
(e.g., almost 2× faster for DeepSeek V3), with comparable quality to
the blueprint method which on average achieved the best metrics. The exception was Qwen3-235B-A22B, which achieved its top results with the baseline blueprint.
Experiments show that the hierarchical method with node filtering provides the best compromise between speed and quality.

\subsection{Qualitative analysis}

\tab{tab:refs}{Comparison of the best and worst english-translated generated summaries}{
  \centering
  \small
  \begin{tabular}{|p{0.15\linewidth}|p{0.75\linewidth}|}
    \hline
    \makecell{\textbf{Title}} & \makecell{\textbf{Text}} \\ \hline

    \makecell{\textbf{A Sound} \\ \textbf{of Thunder}} & \dots
    \textbf{The main character, Eckels}, a thrill-seeking and overconfident hunter, \textbf{pays} a huge sum of money \textbf{for the chance to travel 60 million years back in time to kill a Tyrannosaurus rex}. 
    Before the journey, the guide Travis strictly warns him about the rules: \textbf{under no circumstances should anyone step off the anti-gravity Path} or interfere with the natural course of events, 
    as even the slightest violation could catastrophically change the future.\dots Travis explains \textbf{the fragility of the temporal balance: 
    even the death of a single mouse could wipe out entire species}, and thus alter human history. \textbf{The group tracks down a Tyrannosaurus}, marked with red paint — a sign 
    that \textbf{its death will not affect the future}. However, at the sight of the giant predator \textbf{Eckels panics, steps off the Path, and accidentally crushes a butterfly}\dots 
    Upon returning to 2055 \dots \textbf{the world has changed beyond recognition: the language is coarse, the atmosphere oppressive, and instead of the moderate President Keith, a cruel dictator, Deutscher, is in power.} 
    Eckels realizes that \textbf{his carelessness triggered the “butterfly effect”} — the crushed insect set off a chain of events that distorted history. 
    In despair, he begs to undo the mistake, but \textbf{Travis, understanding the irreversibility of the consequences, raises his rifle.} \dots
    \\ \hline

    \textbf{Kastrjuk} & \dots \uline{The story takes place in a Russian village in early spring, 
    where nature awakens, but people’s lives remain harsh and monotonous.} \textbf{The main character — an old man named Semyon, nicknamed Kastrjuk}, 
    is spending his final days in loneliness, \textbf{tormented by memories of his former strength} and regrets over his present frailty. \textbf{Once he was known as the best worker in the district}, 
    \uline{but now, frail and forgotten, he is forced to stand aside while his fellow villagers work in the fields}. \dots
    \textbf{Only in the evening, persuading his son to let him go to the night watch (to herd horses), does Kastrjuk find brief happiness}. \uline{Out in the open, among the children and under the starry sky, he feels almost young again. 
    By the pond, a mare drinks water reflecting the sunset,} while the \textbf{old man, looking at the Milky Way, whispers a prayer — as if he is reconnecting with the world and regaining lost harmony}. 
    \uline{But it is only a fleeting solace: tomorrow he will again face hopeless loneliness and the realization of his own uselessness.} \dots
    \\ \hline

  \end{tabular}
}

The deviation question-based metrics can be illustrated by results of hierarchical method obtained by DeepSeek V3
Two summaries were chosen for the analysis: "A Sound of Thunder" and "Kastrjuk".
In the first case the model scored high, answering all but one question, but in the other the summary contained answers to only two out of eleven, leading to a low score. 
\tabref{tab:refs} shows the two summaries. For brevity only the main points that affected the final metric were highlighted.
The "Kastrjuk" summary contains many lyrical digressions and stylistic details, making it hard to capture the essence, so the model gets distracted from key facts, 
whereas in "A Sound of Thunder" events are presented sequentially and clearly, with core plot elements explicitly listed, simplifying retrieval of important information.
In the texts, bold marks plot-relevant fragments, while underlines indicate content that could be omitted.




\tab{tab:bl_ref}{Comparison of models in summary generation using the "Blueprint" method (english-translated)}{
  \centering
  \small
  \begin{tabular}{|p{0.15\linewidth}|p{0.75\linewidth}|}
    \hline
    \makecell{\textbf{Model}} & \makecell{\textbf{Text}} \\ 
    \hline

    \makecell{RuadaptQwen3} & "The company *Time Safari* organizes paid excursions into the past for dinosaur hunting, 
    using time machines capable of moving between eras.  
    Clients are required to follow strict rules: to stay on the metal Path \dots
    \\ \hline

    \makecell{tpro} & "In the text, the main character, Eckels, goes on a time safari in order to kill a Tyrannosaurus rex. 
    The company that organizes the safari guarantees only dinosaurs and strictly forbids hunters from stepping 
    off the Path \dots
    Mr. Travis, the safari guide, explains that even the destruction of a single mouse could lead to the extinction of all its descendants \dots
    \\ \hline

    \makecell{DeepSeek V3} & "**Summary by outline:** 1. **Eckels** — the hunter \dots
    2. **The company ‘Time Safari’** organizes hunting in the past \dots
    3. **Travis** — the guide supervising the expedition. \dots
      \\ \hline
  \end{tabular}
}


Comparing model behavior, DeepSeek V3 generally outperforms smaller models; however, within the blueprint method, in 30\% of cases
RuadaptQwen3-32B-Instruct-v2 performs best, and tpro in 43\%. For reference, consider the summary for "A Sound of Thunder" generated with the blueprint method,
with small excerpts shown in \tabref{tab:bl_ref}.
While the DeepSeek V3 summary resembles a numbered list of main events,
the outputs from RuadaptQwen3-32B-Instruct-v2 and tpro are cohesive narratives that cover the key plot points.

\tab{tab:z}{Comparison of hierarchical and blueprint methods}{
  \centering
  \small
  \begin{tabular}{|p{0.15\linewidth}|p{0.75\linewidth}|}
    \hline
    \makecell{\textbf{Method}} & \makecell{\textbf{Text}} \\ \hline

    \textbf{Hierarchical} & \dots Zhulka is a graceful, well-groomed \textbf{horse} that lives on the estate \dots
    \\ \hline

    \textbf{Blueprint} & \dots Zhulka was a small black \textbf{dog} with yellow markings \dots
    \\ \hline

  \end{tabular}
}

Note that the best result overall was achieved by the blueprint method with the large model Qwen3-235B-A22B,
as shown in \tabref{tab:results_models}. For comparison, on the story "Barbos and Zhulka",
the hierarchical method with Qwen3-235B-A22B misclassified "Zhulka" as a horse rather than a dog as shown in \tabref{tab:z}. 
Also, DeepSeek V3 tends to strictly follow the blueprint template and produces a numbered list
of key events and main characters, rather than a coherent summary, whereas Qwen3-235B-A22B writes plain text.
Thus, the unmodified blueprint method delivered the best results when using the strongest available model - Qwen3-235B-A22B.

\subsection{Time measurements}

The results in seconds (average of three runs) are in \tabref{tab:timing}.
They confirm that modifications speed up generation.
To confirm the efficiency of proposed algorithm modifications and measure the actual speed up we conducted an isolated test using text "1408". 
The average results of three runs are provided in \tabref{tab:timing}. 
Interestingly, larger models such as Qwen3-235B-A22B and DeepSeek V3 showed higher speed
than some 32B models achieving almost a 300\% speed up.
A key reason is the Mixture-of-Experts (MoE) architecture:
during generation only a subset of parameters is active (e.g., $\approx30B$ out of $\approx600B$),
thus maintaining throughput of smaller models while having substantially higher level of knowledge and task solving skills.
Moreover, both RuadaptQwen3-32BInstruct-v2 and tpro generate at least 1.5x more tokens, which noticeably increases the overall runtime.

\tab{tab:timing}{Runtime (seconds) for a text of 81{,}049 characters (11 chunks).}{
  \small
\begin{tabular}{lcccc}
\textbf{Model} & \makecell{Hierarchical} & \makecell{Hierarchical\\with node filtering} & \makecell{Blueprint} & \makecell{Blueprint\\with clustering} \\
\hline
DeepSeek V3                  & 237.83 & 72.42 & 292.80 & 268.75 \\
Qwen3-235B-A22B              & 113.24 & 39.45 & 215.63 & 145.20 \\
\hline
RuadaptQwen3-32BInstruct-v2  & 218.23 & 72.54 & 227.7 & 203.30 \\
tpro                         & 472.23 & 127.38 & 391.29 & 185.94 \\
\hline
RuadaptQwen2.5-7B-Lite-Beta  & 84.64 & 25.70 & 103.66 & 78.99 \\
yagpt5lite                   & 34.17 & 14.08 & 99.70 & 27.26 \\
\end{tabular}
}

% ================= CONCLUSION =================

\section*{Conclusion}
In this work we introduced RuBookSum, first open dataset for Russian long-narrative summarization.
To address high computational costs of LLM-based summary generation we proposed two optimizations to existing approaches: 
hierarchical with node filtering and blueprint method with clustering.
The hierarchical method with node filtering achieves up to 300\% speed up with minimal quality loss,
making it a perfect choice for long-document summarization under tight context window limits.

Our comparative analysis shows that larger models such as DeepSeek V3 and Qwen3-235B-A22B generally deliver higher key question coverage and factual agreement
while having more complete summaries than smaller models, especially with hierarchical and blueprint methods.
However, for certain text types and methods (e.g., baseline blueprint), compact models such as RuadaptQwen3-32B-Instruct-v2 can be competitive cost-efficient alternative.
Qualitative analysis shows, that models are better at retelling linear texts with simple descriptions of events, while books with an abundance of lyrical digressions 
lead to models omitting key facts. In addition, the blueprint method gives significantly better retelling using the strong Qwen3-235B-A22B model, 
but some summaries can turn out to be similar to a list of key events, rather than a coherent unified text.

% ================= ACKNOWLEDGEMENTS =================

\section*{Acknowledgements}
The study was supported by grant No. 25-11-00191 from the Russian Science Foundation.
The work was carried out using the supercomputer "MSU-270" of the Lomonosov Moscow State University.

% ================= REFERENCES =================
\openaccess

\bibliographystyle{superfri}
\begin{thebibliography}{99}
\bibitem{BookSum}
BOOKSUM: A Collection of Datasets for Long-form Narrative Summarization / Wojciech Kryscinski, Nazneen Rajani, Divyansh Agarwal et al. // Findings of the Association for Computational Linguistics: EMNLP 2022 / Ed. by Yoav Goldberg, Zornitsa Kozareva, Yue Zhang. - Abu Dhabi, United Arab Emirates: Association for Computational Linguistics, 2022. - . - Pp. 6536-6558. https://aclanthology.org/2022.findings-emnlp.488/.

\bibitem{alexandria}
Echoes from Alexandria: A Large Resource for Multilingual Book Summarization / Alessandro Scir`e, Simone Conia, Simone Ciciliano, Roberto Navigli // Findings of the Association for Computational Linguistics: ACL 2023 / Ed. by Anna Rogers, Jordan Boyd-Graber, Naoaki Okazaki. - Toronto, Canada: Association for Computational Linguistics, 2023. - . - Pp. 853-867. https://aclanthology.org/2023.findings-acl.54/.

\bibitem{fables}
FABLES: Evaluating faithfulness and content selection in book-length summarization \\/ Yekyung Kim, Yapei Chang, Marzena Karpinska et al. // First Conference on Language Modeling. - 2024. https://openreview.net/forum?id=YfHxQSoaWU.

\bibitem{Briefly}
\textit{"Narodny Briefly".}  
Digital library of short summaries of literary works.
\url{https://wiki.briefly.ru/} (accessed: 30.07.2025).

\bibitem{librusec}
Library of works of art.
\url{https://librusec.org//} (accessed: 30.07.2025).

\bibitem{hierarchical} 
Wu, Jeff, et al. "Recursively summarizing books with human feedback." arXiv preprint arXiv:2109.10862 (2021).

\bibitem{blueprint} 
Text-Blueprint: An Interactive Platform for Plan-based Conditional Generation / Fantine Huot, Joshua Maynez, Shashi Narayan et al. // Proceedings of the 17th Conference of the European Chapter of the Association for Computational Linguistics:
System Demonstrations / Ed. by Danilo Croce, Luca Soldaini. - Dubrovnik, Croatia: Association for Computational Linguistics, 2023. - . - Pp. 105-116. https:
//aclanthology.org/2023.eacl-demo.13/.

\bibitem{rouge}
\textit{ROUGE.}
Lin, Chin-Yew. "Rouge: A package for automatic evaluation of summaries." Text summarization branches out. 2004.

\bibitem{bertscore}
\textit{BERTScore.}
BUCKLEY C. Evaluating Evaluation Measure Stability //ACM SIGIR 2000 Proceedings. - 2000.

\bibitem{qwen3}
\textit{Qwen3-235B.}
Yang A. et al. Qwen3 technical report //arXiv preprint arXiv:2505.09388. - 2025.

\bibitem{ruadapt}
\textit{RuadaptQwen.}
Tikhomirov, Mikhail, and Daniil Chernyshev. "Facilitating large language model russian adaptation with learned embedding propagation." Journal of Language and Education 10.4 (40) (2024): 130-145.

\bibitem{deepseek}
\textit{DeepSeek V3.}
Liu A. et al. DeepSeek-V3 Technical Report //CoRR. - 2024.

\bibitem{tpro}
T-Bank has opened access to its own Russian-language language model in the weight category of 7-8 billion parameters / T-Bank URL: \url{https://www.tbank.ru/about/news/20072024-t-bank-opened-access-its-own-russian-language-language-model-weight-category-of-7-8-billion-parameters/} (accessed: 10.05.2025).

\bibitem{yagpt}
YandexGPT 5 with reasoning mode // Yandex URL: https://ya.ru/ai/gpt (accessed: 30.07.2025).

\end{thebibliography}

\end{document}
